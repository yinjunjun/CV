%!TEX TS-program = xelatex
%!TEX encoding = UTF-8 Unicode

\documentclass[11pt, a4paper]{article}
\usepackage{fontspec} 

% DOCUMENT LAYOUT
\usepackage{geometry} 
\geometry{a4paper, textwidth=5.5in, textheight=8.5in, marginparsep=7pt, marginparwidth=.6in}
\setlength\parindent{0in}

% FONTS
\usepackage[usenames,dvipsnames]{xcolor}
\usepackage{xunicode}
\usepackage{xltxtra}
\defaultfontfeatures{Mapping=tex-text}
%\setromanfont [Ligatures={Common}, Numbers={OldStyle}, Variant=01]{Linux Libertine O}
%\setmonofont[Scale=0.8]{Monaco}
%%% modified by Karol Kozioł for ShareLaTeX use
\setmainfont[
  Ligatures={Common}, Numbers={OldStyle}, Variant=01,
  BoldFont=LinLibertine_RB.otf,
  ItalicFont=LinLibertine_RI.otf,
  BoldItalicFont=LinLibertine_RBI.otf
]{LinLibertine_R.otf}
\setmonofont[Scale=0.9]{DejaVuSansMono.ttf}

% ---- CUSTOM COMMANDS
\chardef\&="E050
\newcommand{\html}[1]{\href{#1}{\scriptsize\textsc{[html]}}}
\newcommand{\pdf}[1]{\href{#1}{\scriptsize\textsc{[pdf]}}}
\newcommand{\doi}[1]{\href{#1}{\scriptsize\textsc{[doi]}}}
% ---- MARGIN YEARS
\usepackage{marginnote}
\newcommand{\amper{}}{\chardef\amper="E0BD }
\newcommand{\years}[1]{\marginnote{\scriptsize #1}}
\renewcommand*{\raggedleftmarginnote}{}
\setlength{\marginparsep}{11pt}
\reversemarginpar

% HEADINGS
\usepackage{sectsty} 
\usepackage{multicol} 
\usepackage[normalem]{ulem} 
\sectionfont{\mdseries\upshape\Large}
\subsectionfont{\mdseries\scshape\normalsize} 
\subsubsectionfont{\mdseries\upshape\large} 

% PDF SETUP
% ---- FILL IN HERE THE DOC TITLE AND AUTHOR
\usepackage[%dvipdfm, 
bookmarks, colorlinks, breaklinks, 
% ---- FILL IN HERE THE TITLE AND AUTHOR
	pdftitle={Junjun Yin - vita},
	pdfauthor={Junjun Yin},
	pdfproducer={}
]{hyperref}  
\hypersetup{linkcolor=blue,citecolor=blue,filecolor=black,urlcolor=MidnightBlue} 

% DOCUMENT
\begin{document}
{\LARGE Junjun Yin, Ph.D}
\begin{center}
\line(1,0){450}
\end{center}
\section*{Contact Information}

\begin{multicols}{2}
 \begin{flushleft}
Room 813, Oswald Tower\\
The Pennsylvania State University\\
University Park, PA, 16802, United States
\end{flushleft}
\columnbreak
\begin{flushright}
Phone: +1 217-819-6366\\[.1cm]
Email: \href{mailto:a.jyin@psu.edu}{jyin@psu.edu}\\[.1cm]
\textsc{url}: \href{https://yinjunjun.github.io}{https://yinjunjun.github.io}
\end{flushright}
\end{multicols}


%%\hrule
\section*{Current Position}
\begin{multicols}{2}
 \begin{flushleft}
\emph{Assistant Research Professor}\\
Social Science Research Institute\\
The Pennsylvania State University\\
\end{flushleft}
\columnbreak
\begin{flushright}
\emph{ICDS Associate}\\
Institute for Computational and Data Sciences\\
\end{flushright}
\end{multicols}
%%\hrule

\section*{Previous Position}
\emph{Postdoctoral Research Fellow}\\
CyberGIS Center for Advanced Digital and Spatial Studies\\
Department of Geography and Geographic Information Science\\
National Center for Supercomputing Applications (NCSA)\\
University of Illinois at Urbana-Champaign, IL, USA
%%\hrule
\section*{Research Interests}
GIScience; Spatial Analysis and Modeling; Urban Informatics; Computational Geography;
Geo-Complexity and Human Mobility; (Geo)Visual-Analytics and Data Mining; 
GeoSpatial Big Data; GeoComputing; Spatial Interaction; High-performance (cyberGIS), Web and Mobile GIS

%\hrule
\section*{Education}
\noindent
\years{2009 - 2013}\textit{Ph.D} in Spatial Information Science, Dublin Institute of Technology, Ireland\\
Scince Fundation Ireland Scholarship\\
\textit{Advisors: Dr. James D. Carswell (Dublin Institute of Technology), Dr. Michela Bertolotto (University College Dublin)}\\
\years{2006 - 2009}\textit{MSc} in Geomatics, University of Gävle, Sweden\\
\textit{Advisor: Dr. Bin Jiang (University of Gävle)}\\
\years{2002 - 2006}\textit{BSc} in Electronic Engineering, University of Electronic Science and Technology of China

%%\hrule
\section*{Academic Experience}
\noindent
\years{2017-}Social Science Big Data Research Scientist, Pennsylvania State University, USA\\
\years{2016} Lecturer, Department of Geography and Geographic Information Science; University of Illinois at Urbana-Champaign, USA\\
\years{2014-2016} Postdoctoral Research Associate, CyberGIS Center for Advanced Digital and Spatial Studies; National Center for Supercomputing Applications; University of Illinois at Urbana-Champaign, USA\\
\years{2013-2014} Assistant Lecturer, Dublin Institute of Technology, Ireland\\
\years{2007-2008} Research Assistant, Hong Kong Polytechnic University, Hong Kong SAR, China

\section*{Research Grants and Projects}
\textbf{Co-PI}. ``Are we more willing to speak up through mobile phones?  A comparison of desktop vs. mobile political sharing on Facebook''. Social Science Research Council, $\$$49,017. Project dates: 02/01/2020—01/31-2021\\
\textbf{Senior Personnel}. ``Pursuing Opportunities for Long-term Arctic Resilience for Infrastructure and Society''. NSF (National Science Foundation) NNA grant, $\$$3,000,000. Project dates: 01/01/2020—12/31-2022\\
\textbf{Co-PI}. ``RR: The Generalizability and Replicability of Twitter Data for Population Research''. NSF (National Science Foundation) SOC grant, $\$$500,000. Project dates: 07/15/2018—06/30-2021\\
\textbf{Co-PI}. ``Understanding National Park Visitors’ Spatial Behavior with Twitter Data''. Penn State Seed Grant (SSRI, ICS, and IST), $\$$20,000. Project dates: 03/01/2019—02/28-2021\\
\textbf{PI}. ``Data Science: Mining sequential mobility patterns from semantic Twitter user trajectories''. Microsoft Azure Data Science Research Award, $\$10,000$. Project dates: 05/01/2018—04/30/2019\\
\textbf{Co-PI}. ``An Innovative Approach to Tackle the Opioid Epidemic: Utilizing Twitter Data and Integrating Big Data Analytics and Spatial and Social Network Analyses'. Social Science Research Institute Level 2 Research Award, $\$$20,000. Project dates: 09/01/2018—08/30-2020\\
\textbf{Co-I}. ``Cognitive Changes Associated with Hormonal Treatment for Breast Cancer''. Health $\&$ Environment initiative seed grant award, $\$$49,972. Project dates: 07/01/2017—06/30-2019\\
\textbf{Co-PI}. ``The Generalizability and Replicability of Twitter Data for Population Research''. ICS seed grant award, $\$$35,000. Project dates: 07/01/2018—06/30-2019\\
\textbf{Co-PI}. ``Ecological Migration in a Large-Scale Quasi-Experiment Design in China: Implications of Climate Change, Landscape Structure, Ecosystem Services and Government Intervention''. IEEE seed grant award, $\$$25,000. Project dates: 04/15/2017—06/30-2018\\
\textbf{PI}. ``Mining Twitter User Demographics as a First-Step in Big Data for Population Research''. XSEDE (Extreme Science and Engineering Discovery Environment) Startup computational resource allocation award, 50,000 SUs $\&$ 40 TB storage (estimated $\$$2,794.78). Project dates: 03/27/2017—03/26-2019\\
\textbf{PI}. ``A cloud computing enabled GIS platform for the integration and synthesis of multi-layer geospatial data sources in urban studies: Understanding urban dynamics from geospatial Big Data''. Microsoft Azure Data Science Research Award, $\$20,000$. Project dates: 12/04/2016—12/03/2017\\
\textbf{SP/Research Scientist}. ``Activity space contexts and measuring environmental exposure in behavioral research''. R21 NCI grant (ASCMEE Study).

\section*{Publications}
%\subsection*{Work in progress}
%\years{2019}\textbf{Yin, J.} and Chi, G. (2017). A tale of three cities: A GIS-based data synthesis approach to integrating geographic context for understanding spatiotemporal activities in the urban environment. \textit{manuscript ready}
\subsection*{Submitted}
\years{2019} Abdar, M., Basiri, M.E., \textbf{Yin, J.}, Asadi, S. and Chi, G., Energy Choices in Alaska: Mining People's Perception and Attitudes from Geotagged Tweets. \textit{revision to Renewable $\&$ Sustainable Energy Reviews}\\
\years{2019} \textbf{Yin, J.}, Gao, Y.. Chi, G. and Van Hook, J., An Evaluation of Geo-located Twitter Data for Measuring Human Migration. \textit{submitted to Annals of the Association of American Geographers}\\
\years{2019} Zheng, K., \textbf{Yin, J.}, Liu, M., Zhang, M. and Xie, M., A joint neural network model for geographical entity relation extraction from Chinese texts. \textit{submitted to Journal of Information Science}\\
\years{2019}\textbf{Yin, J.} and  Chi, G., Characterizing People’s Daily Activity Patterns in the Urban Environment: A mobility network approach with geographic context-aware Twitter data. \textit{submitted to International Journal of Geographical Information Science}
\subsection*{Peer Reviewed Journals}
\noindent
\years{2019} Pu, Y., Zhao, X., Chi, G., Zhao, S., Wang, J., Jin, Z. and \textbf{Yin, J.} (2019). Design and implementation of a parallel geographically weighted k-nearest neighbor classifier. \textit{Computers $\&$ Geosciences}, 127, pp. 111-122\\
\years{2018}Gao, Y., Wang, S. Padmanabhan, A., \textbf{Yin, J.} and Cao, G. (2018). Mapping Spatiotemporal Patterns of Events Using Social Media: A Case Study of Influenza Trends. \textit{International Journal of Geographical Information Science}, 32(3), pp. 425-449\\
\years{2017}\textbf{Yin, J.}, Soliman, A., Yin, D. and Wang, S. (2017). Delineate Urban Boundaries in Great Britain from the Network of Large Scale Twitter User Spatial Interactions. \textit{International Journal of Geographical Information Science}, 31(7), pp. 1293-1313\\
\years{2017} Soliman, A., Soltani, Q., \textbf{Yin, J.}, Padmanabhan, A., and Wang, S (2017). Social sensing of urban land use based on analysis of Twitter users' mobility patterns. \textit{PLoS ONE}, 12(7): e0181657. DOI:10.1371/journal.pone.0181657\\
\years{2017}Zheng, K., Kwan, M.P., Fang, L., \textbf{Yin, J.}, Gu, D. and Fu, Y. (2017). A Topology-concerned Spatial Vector Data Model for Column-oriented Databases. \textit{International Journal of Database Theory and Application}, 10(5), pp. 33-46\\
\years{2016}\textbf{Yin, J.}, Gao, Y., Du, Z. and Wang, S. (2016). Exploring Multi-Scale Spatiotemporal Twitter User Mobility Patterns with a Visual-Analytics Approach. \textit{ISPRS International Journal of Geo-Information}, 5(10):187.\\
\years{2016}Jiang, B., Ma, D., \textbf{Yin, J.} and Sandberg, M. (2016). Spatial Distribution of Tweet Numbers and Densities in Cities. \textit{Geographical Analysis}, 48(3), pp. 337-351\\
\years{2015}Jiang, B.,\textbf{Yin, J.} and Liu, Q. (2015). Zipf's Law for All the Natural Cities around the World. \textit{International Journal of Geographical Information Science, 29(3), pp. 498-522}\\
\years{2014}Jiang, B. and \textbf{Yin, J}. (2014). Ht-Index to Quantify the Fractal or Scaling Structure of Geographic Features. \textit{Annals of the Association of American Geographers}, 104(3), pp. 530–540\\
\years{2013}\textbf{Yin, J.} and Carswell, J.D. (2013). Spatial Search Techniques for Mobile 3D Queries in Sensor Web Environments. \textit{ISPRS International Journal of Geo-Information}, 2(1): pp.135-154\\
\years{2010} Carswell, J. D., \textbf{Yin, J.} and Gardiner, K. (2010). 3DQ: Threat Dome Visibility Querying on Mobile Devices \textit{GIM International}, 24(8)\\
\years{2010}Carswell, J.D., Gardiner, K. and \textbf{Yin, J.} (2010). Mobile Visibility Querying for LBS. \textit{Transactions in GIS}, 14(6): pp. 791-809, Wiley online library\\
\years{2009}Jiang, B., \textbf{Yin, J.} and Zhao, S. (2009). Characterizing the human mobility pattern in a large street network. \textit{Physical Review E}, 80(2), 021136\\
\years{2008}Jiang, B., Zhao, S. and \textbf{Yin, J.} (2008). Self-organized natural roads for predicting traffic flow: a sensitivity study. \textit{Journal of statistical mechanics: Theory and experiment}, P07008, IOP Publishing

\subsection*{Peer Reviewed Conferences and Lecture Notes}
\noindent
\years{2019}\textbf{Yin, J.}, and Chi, G. (2019). Understanding Spatiotemporal Urban Activity Patterns with Geo-located Twitter data: A GIS-based Synthesis Approach. \textit{66th Annual North American Meetings of the Regional Science Association International (NARSC)}, November 13-16, 2019, Pittsburg, Pennsylvania\\
\years{2019}\textbf{Yin, J.}, and Chi, G. (2019). An Evaluation of Geo-located Twitter Data as Indicators for Human Migration. \textit{IUSSP Research Workshop on Digital Demography in the Era of Big Data}, June 6-7, 2019, Seville, Spain\\
\years{2019}Chi, G., \textbf{Yin, J.}, Van Hook, J., Plutzer, E. and Heng, X. (2019). The Generalizability of Twitter Data for Population Research. \textit{Population Association of America (PAA) Annual Meeting}, April 22-25, Washington, DC, USA\\
\years{2018}\textbf{Yin, J.}, Chi, G. and Van Hook, J. (2018). Evaluating the Representativeness in the Geographic Distribution of Twitter User Population. \textit{The 12th Workshop on Geographic Information Retrieval in ACM SIGSPATIAL 2018}, November 6-9, 2018, Seattle, Washington, USA\\
\years{2018}Jeong, M., \textbf{Yin, J.}, and Wang, S. (2018). Outliers Detection and Comparison of Origin-Destination Flows with Data Depth. \textit{The 10th International Conference on Geographic Information Science (GIScience 2018)}, August 28-31, 2018, Melbourne, Australia\\
\years{2017}\textbf{Yin, J.} (2017). Mining sequential mobility pattern from semantics enriched Twitter user trajectories. \textit{NSF Mobility Workshop on Analyzing Movement and Mobility within Geographic Context}, May 11-12, 2017, The Ohio State University, Columbus, Ohio\\
\years{2017}Chi, G., \textbf{Yin, J.} and Hook, J.V. (2017). Predicting Twitter User Demographics as a First Step in Big Data for Population Research. \textit{The 28th International Population Conference of the International Union for the Scientific Study of Population}, October 29-November 4, 2017, Cape Town, South Africa\\
\years{2016}\textbf{Yin, J.}, Lu, B., Yin, D. and Wang, S. (2016). A scalable visual-analytics approach for studying mobility networks: Revealing hierarchical structures in taxi mobility flows of New York, \textit{The Third International Conference on CyberGIS and Geospatial Data Science}, July 26-28, 2016, Urbana, Illinois\\
\years{2016}\textbf{Yin, J.}, Gao, Y. and Wang, S. (2016). Urban Sensing from Volunteered Citizen Participation using Mobile Devices. In \emph{Seeing Cities through Big Data: Research, Methods and Applications in Urban Informatics}, Springer\\
\years{2015}Soliman, A., \textbf{Yin, J}, Soltani, K., Padmanabhan, A., and Wang, S. (2015). Where Chicagoans tweet the most: Semantic analysis of preferential return locations of Twitter users, \textit{1st International Workshop on Smart Cities and Urban Analytics 2015, 23rd ACM SIGSPATIAL International Conference on Advances in Geographic Information Systems}\\ 
\years{2014}\textbf{Yin, J.} and Wang, S. (2014). Understanding the Evolvements of Natural Cities from Nighttime Light Images: A CyberGIS-Enhanced Approach to Large Scale Geospatial Data Analysis, \textit{The Second International Conference on CyberGIS and Geodesign}, August 19-21, 2014, Redlands, California\\
\years{2014}\textbf{Yin, J.}, Gao, Y. and Wang, S. (2014). CyberGIS Enabled Urban Sensing from Volunteered Citizen Participation using Mobile Devices. \textit{NSF Workshop on Big Data and Urban Informatics 2014}\\
\years{2013}Truong-Hong, L., Thi, T.T.P, \textbf{Yin, J.} and Carswell, J.D. (2013). Detailed 3D building models for Google Earth integration. In \textit{Proceedings of the 13th International Conference on Computational Science and Its Applications (ICCSA 2013)}, Ho Chi Minh City, Vietnam: Springer (\textbf{\emph{$*$Best paper award}})\\
\years{2013}Thi, T.T.P, Truong-Hong, L., \textbf{Yin, J.} and Carswell, J.D. (2013). Exploring Spatial Business Data: A ROA based eCampus application. In \textit{Proceedings of the 11th International conference on Web and Wireless Geographical Information Systems}, Banff, AB: Springer Berlin Heidelberg, pp. 164-179\\
\years{2012}\textbf{Yin, J.} and Carswell, J.D. (2012). Effects of Variations in 3D Spatial Search Techniques on Mobile Query Speed vs Accuracy. \textit{Web and Wireless Geographical Information Systems}, Naples, Italy. (\textbf{\emph{$*$Best paper award}})\\
\years{2012}Carswell, J.D. and \textbf{Yin, J.} (2012). Mobile Spatial Interaction in the Future Internet of Things. In \textit{Proceedings of the 20th International Conference on Geoinformatics (GEOINFORMATICS)}, 2012, pp. 1-6\\
\years{2012}\textbf{Yin, J.} and Carswell, J.D. (2012). MobiSpatial: Open-source for Mobile Spatial Interaction. \textit{Proceedings of the 27th Annual ACM Symposium on Applied Computing}, pp. 572-573\\
\years{2012}\textbf{Yin, J.} and Carswell, J.D. (2011). Touch2Query enabled mobile devices: A case study using OpenStreetMap and iPhone. \textit{Web and Wireless Geographical Information Systems}, pp. 203-218, Springer\\
\years{2009} Gardiner, K., \textbf{Yin, J.} and Carswell, J.D. (2009). EgoViz: A mobile based spatial interaction system. \textit{Web and Wireless Geographical Information Systems}, pp. 135-152, Springer
\subsection*{Books}
\years{2009} \textbf{Yin, J.} (2009), \emph{The Topological Patterns of Urban Street Networks: Exploring the topological patterns of urban street networks from analytical and visual perspectives}, VDM: Germany, ISBN: 3639161734
\subsection*{Invited talks}
\years{2019}Yin, J. (2019). Spatial Networks: A computational geography approach to new insights into spatial interactions and geo-complexity. Jan, 2019, University of Florida, Gainesville, FL , USA\\
\years{2018}Yin, J. (2018). Computational Geography for Capturing Geo-Complexity in Urban Studies. June, 2018, Newcastle University, Newcastle upon Tyne, UK\\
\years{2018}Yin, J. (2018). Advanced Methods and Techniques for Big GeoData. June, 2018, ITC, University of Twente, Enschede, the Netherlands\\
\years{2018}Yin, J. (2018). Geo-Complexity and Human mobility: Through the lens of spatial Big Data to understand urban dynamics. February, 2018, University of Denver, Denver, USA\\
\years{2018}Yin, J. (2018). High-performance Computing with Hadoop. Software in the Humanities and Social Sciences Workshop. February, 2018, Penn State, University Park, USA
\subsection*{Presentations}
\years{2018}Yin, J. (2018). Spatial Interaction Patterns of Preferential Return Behaviors in People’s Daily Life. The Association of American Geographers Annual Meeting, April 10-14, 2018, New Orleans, Louisiana, USA\\
\years{2017}Yin, J. (2017). A mobility network approach to modeling urban spatial interactions: Insights from the movement Big Data in New York City. The Association of American Geographers Annual Meeting, April 5-9, 2016, Boston, Massachusetts, USA\\
\years{2016}Yin, J. and Wang, S. (2016). Mining Mobility Patterns From Semantic Twitter User Trajectories. The Association of American Geographers Annual Meeting, March 29-April 2, 2016, San Francisco, California, USA\\
\years{2015}Yin, J. and Wang, S. (2015). Finding community structures of UK cities based on large-scale Twitter user mobility patterns, \textit{Association of American Geographers Annual Meeting, Chicago, 2015}, April 21-25, 2015, Chicago, Illinois, USA\\
\years{2011}Yin, J. (2011). Web-service based Mobile Geospatial Application Development using Python, \textit{PyCon 2011}, Dublin, Oct 8-9

\section*{Teaching Experience as Instructor}
\years{2016 - } GEOG 479: Advanced Topics in GIS -- CyberGIS, Spring, 2016, \textit{Lecturer}, Department of Geography and Geographic Information Science, UIUC\\
\years{2015 - } Parallel Databases, \textit{workshop/short course series}, CyberGIS Center, UIUC\\
\years{2015 - } Interactive Visualization of Large-scale Movement Data using Apache Spark, \textit{Summer school series}, CyberGIS Center, UIUC\\
\years{2015 - } Introducing the CyberGIS Toolkit in high performance computing environment, \textit{workshop/short course series}, CyberGIS Center, UIUC\\
\years{2015 - } Taming with geospatial Big Data with Hadoop, \textit{workshop/short course series}, CyberGIS Center, UIUC\\
\years{2014 - } Getting to know CyberGIS, \textit{workshop/short course series}, CyberGIS Center, UIUC\\
\years{2014 - 2016} Education and outreach at the CyberGIS Commons, \textit{Lead}, University of Illinois at Urbana-Champaign, 2014 - 2016\\
\years{2013} Programming in C, \textit{assistant lecturer}, Dublin Institute of Technology, 2013 - 2014\\
\years{2013} Object oriented programing for game development, \textit{senior demonstrator/assistant lecturer}, Dublin Institute of Technology, 2013 - 2014\\
\years{2013} User interface design and GUI programing with Java, \textit{senior demonstrator/assistant lecturer}, Dublin Institute of Technology, 2013 - 2014

%\hrule
\section*{Skills}
Programing languages:Java, Python, Objective-C, Matlab, R, JavaScript, NoSQL\\
High performance computing: Apache Hadoop, Apache Spark, MongoDB\\
Mobile and Web development: iOS and Android, Cesium 3D Globe, D3.js\\
Spatial databases: PostGIS, Oracle Spatial, SpatiaLite, ArcGIS

%\hrule
\section*{Honors and Awards}
\noindent
\years{2013}\emph{Fiosraigh Head of School Research Award}: for excellence in research, Dublin Institute of Technology\\
\years{2013}\emph{EU Future Internet Award}: for excellence in Future Internet research\\
\years{2009}\emph{PhD scholarship from Science Foundation Ireland (SFI)}: StratAG PhD scholarship\\
\years{2006}\emph{Outstanding Graduate Student Award}: University of Electronic Science and Technology of China\\
\years{2006}\emph{Best Undergraduate Dissertation Award}: Department of Electronic Engineering, University of Electronic Science and Technology of China\\
\years{2005}\emph{National Undergraduate Electronic Design Contest: Second prize}:Embedded system for Multi-channel and frequency wave generating, (Team: Junjun Yin, Yu Chen, and Xia Yu)

%\hrule
\section*{Service to the Profession}
\subsection*{Guest editors}
Guest editor for Special Issue of ``Chinese Sociological Dialogue``: Spatiotemporal Big Data and Sustainable Social Development (2017)
\subsection*{Program Committees and Organizers}
\begin{itemize}
\item{\textbf{Chair} for Innovations in Spatial Data Analysis and Modeling, 66th Annual North American Meetings of the Regional Science Association International (NARSC), Pittsburg, Pennsylvania, November 13-16, 2019}
\item{\textbf{PC} member for W2GIS 2019: 17th International Symposium on Web and Wireless Geographical Information Systems, May 16-17, 2019, Kyoto, Japan}
\item{\textbf{PC} member for International Conference on Location-based Social Media Data, March 5-7, 2015, Athens, Georgia, USA}
\item{\textbf{Co-chair} for Advances in Spatial Interaction Models and Methods in the Big Data Era I, Spatiotemporal Symposium, 2018 AAG Annual Meeting, New Orleans, April 10 – April 14, 2018}
\item{\textbf{Chair} for Big Movement Data for Geospatial Analytics on Urban Interactions, Symposium on Human Dynamics in Smart and Connected Communities, 2017 AAG Annual Meeting, Boston, Massachusetts, April 5 - April 9, 2017}
\item{\textbf{Chair} for Understanding Urban Dynamics Based on Movement Big Data, 2016 AAG Annual Meeting, San Francisco, California, March 29 - April 2, 2016}
\end{itemize}
\subsection*{Reviewer for Proposals}
\begin{itemize}
  \setlength\itemsep{0em}
  \item \emph{Methodology, Measurement, and Statistics Program, \textbf{National Science Foundation}} (2019)  
\end{itemize} 


\subsection*{Reviewer for Journals}
Has reviewed for \textbf{23} journals
\begin{itemize}
  \setlength\itemsep{0em}
  \item \emph{Annals of the American Association of Geographers} (2018-2019)  
  \item \emph{International Journal of Geographical Information Science (IJGIS)} (2010, 2014-2018)
  \item \emph{Journal of Geovisualization and Spatial Analysis} (2018)
  \item \emph{Cartography and Geographic Information Science} (2018)
  \item \emph{Computers, Environment and Urban Systems (CEUS)} (2009-2010, 2013-2019)
  \item \emph{Journal of Urban Technology} (2019)
  \item \emph{Transactions on Intelligent Transportation Systems} (2019)
  \item \emph{EPJ Data Science} (2017)
  \item \emph{Geomatica} (2018)
  \item \emph{GeoJournal} (2018)
  \item \emph{Chinese Sociological Dialogue} (2018)
  \item \emph{Spatial Information Research} (2018)
  \item \emph{Sensors} (2018)
  \item \emph{Geographical Analysis} (2016-2018)
  \item \emph{Cluster Computing} (2017)
  \item \emph{Journal of Geographical Systems} (2016-2017)
  \item \emph{Demography} (2016-2019)
  \item \emph{Environment and Planning B: Planning and Design} (2016-2017)
  \item \emph{Science China Information Science} (2015)
  \item \emph{PLOS One} (2013, 2015, 2017)
  \item \emph{Cities} (2014)
  \item \emph{ISPRS International Journal of Geo-Information} (2013-2015)
  \item \emph{Journal of Location Based Services} (2013)
\end{itemize} 

\subsection*{Reviewer for Conferences}
\begin{itemize}
  \setlength\itemsep{0em}
  \item \emph{GIScience} (2014, 2016)
  \item \emph {UbiComp} (2016)
  \item \emph {W2GIS} (2019)
\end{itemize}

\section*{Referees}
\noindent
\textbf{Dr. Bin Jiang}\\
Professor\\
Department of Technology and Built Environment\\
University of Gävle, SE-801 76\\
Gävle, Sweden, Office: 11:232\\
Email: bin.jiang@hig.se\\
Phone: +46 26 64 8901\\

\textbf{Dr. James D. Carswell}\\
Principal Investigator\\
Head of Spatial Information Technologies Research\\
Digital Media Centre\\
Dublin Institute of Technology, DIT Aungier Street, Dublin 2, Ireland\\
Email: james.carswell@tudublin.ie\\
Phone: +353 (1) 402 3264\\

\textbf{Dr. Shaowen Wang}\\
Professor and Department Head\\
Richard and Margaret Romano Professorial Scholar\\
CyberGIS Center for Advanced Digital and Spatial Studies\\
CyberInfrastructure and Geospatial Information Laboratory\\
Department of Geography and Geographic Information Science\\
University of Illinois at Urbana-Champaign, Urbana, IL 61801, USA\\
Email: shaowen@illinois.edu\\
Phone: +1 217 333 7608\\

\textbf{Dr. Guangqing Chi}\\
Associate Professor\\
Department of Agricultural Economics, Sociology, and Education (112E Armsby)\\
Population Research Institute (803 Oswald)\\
The Pennsylvania State University, University Park, PA 16802\\
Email: gchi@psu.edu\\
Phone: +1 814-826-4686
\vfill{}

\begin{center}
{\scriptsize  Last updated: \today\- •\- 
% ---- PLEASE LEAVE THIS BACKLINK FOR ATTRIBUTION AS PER CC-LICENSE
%Available from \href{http://sites.google.com/site/yinjunjun/}{
%\fontspec{Times New Roman}
%\XeTeX }\\
% ---- FILL IN THE FULL URL TO YOUR CV HERE
\href{https://yinjunjun.github.io}{Junjun Yin}}
\end{center}

\end{document}